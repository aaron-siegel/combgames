\ProvidesFile{cg-tree.tex}

%%%%%%%%%%%%%%%%%%%%%%%%%%%%%%%%%%%%%%%%%%%%%%%%%%%%%%%%%%%%%%
%%                                                          %%
%%                      Game Trees                          %%
%%                                                          %%
%%%%%%%%%%%%%%%%%%%%%%%%%%%%%%%%%%%%%%%%%%%%%%%%%%%%%%%%%%%%%%

% Some rather complicated logic for an analogue of \futurelet that skips over spaces:

\def\cg@fletnosp#1#2{\def\@tempa{#1}\def\@tempb{#2}\futurelet\@tempc\cg@fletnosp@}
\def\cg@fletnosp@{\ifx\@tempc\@sptoken\let\@tempd\cg@fletnosp@@\else\expandafter\let\@tempa\@tempc\let\@tempd\@tempb\fi\@tempd}
{\def\:{\cg@fletnosp@@} \expandafter\gdef\: {\futurelet\@tempc\cg@fletnosp@}}

\define@key[cgk]{cgtree}{arrows}{\def\cg@tree@arrows{#1}}
\define@key[cgk]{cgtree}{unit}{\def\cg@tree@unit{#1}}
\define@key[cgk]{cgtree}{nodesep}{\def\cg@tree@nodesep{#1}}
\define@boolkey[cgk]{cgtree}{labels}[false]{}

\define@key[cgk]{cgnode}{arrow}{\def\cg@node@arrow{#1}}
\define@key[cgk]{cgnode}{sep}{\def\cg@node@sep{#1}}
\define@key[cgk]{cgnode}{name}{\def\cg@node@name{#1}}
\define@boolkey[cgk]{cgnode}{ko}[true]{}
\define@boolkey[cgk]{cgnode}{pass}[true]{}

\newcounter{cg@nextnodeord}  % Global node ordinal counter for default names

\newcommand\cgtree[2][]{{
  \setkeys[cgk]{cgtree}{arrows=-,nodesep=.75ex,unit=1cm,labels=false}%
  \setkeys[cgk]{cgtree}{#1}%
  %
  % First pass to calculate boundaries
  \cg@gtinit%
  \let\cg@gtdonode\cg@gtcalcmax%
  \let\cg@gtdoline\relax%
  \cg@gtprocessnode#2\cg@gtend%
  %
  % Second pass to draw!
  \cg@gtinit%
  \let\cg@gtdonode\cg@gtshow%
  \let\cg@gtdoline\cg@gtdrawline%
  \begin{tikzpicture}[x=\cg@tree@unit,y=\cg@tree@unit]
  \cg@gtprocessnode#2\cg@gtend%
  \end{tikzpicture}
}}
\newcommand\cgtreet[2][]{{%
  \setkeys[cgk]{cgtree}{arrows=->,nodesep=.75ex,unit=1cm,labels=false}%
  \setkeys[cgk]{cgtree}{#1}%
  %
  % First pass to calculate boundaries
  \setcounter{cg@gtxmin}{0}\setcounter{cg@gtxmax}{0}\setcounter{cg@gtymin}{0}\setcounter{cg@gtymax}{0}%
  \cg@gtinit%
  \let\cg@gtdonode\cg@gtcalcmax%
  \let\cg@gtdoline\relax%
  \cg@gtprocessnode#2\cg@gtend%
  %
  % Second pass to draw!
  \cg@gtinit%
  \let\cg@gtdonode\cg@gtshow%
  \let\cg@gtdoline\cg@gtdrawline%
  \setlength{\cg@gtxminlen}{\thecg@gtxmin\psxunit-0.5\psxunit}%
  \setlength{\cg@gtxmaxlen}{\thecg@gtxmax\psxunit+0.5\psxunit}%
  \setlength{\cg@gtyminlen}{\thecg@gtymin\psyunit-0.5\psyunit}%
  \setlength{\cg@gtymaxlen}{\thecg@gtymax\psyunit+0.5\psyunit}%
  \begin{pspicture}[shift=\cg@gtyminlen](\cg@gtxminlen,\cg@gtyminlen)(\cg@gtxmaxlen,\cg@gtymaxlen)
  \cg@gtprocessnode#2\cg@gtend%
  \end{pspicture}%
}}

\newcounter{cg@gtxmin}
\newcounter{cg@gtxmax}
\newcounter{cg@gtymin}
\newcounter{cg@gtymax}
\newlength{\cg@gtxminlen}
\newlength{\cg@gtyminlen}
\newlength{\cg@gtxmaxlen}
\newlength{\cg@gtymaxlen}
\newcount\cg@gtx
\newcount\cg@gty
\newcount\cg@gtnextx
\newcount\cg@gtnexty
\newcount\cg@gtparentx
\newcount\cg@gtparenty
\newcount\cg@gtxadjustment
\newlength\cg@gthalfunit

\def\cg@gtinit{
  \cg@gtx=0
  \cg@gty=0
  \cg@gtnextx=0
  \cg@gtnexty=0
  \cg@gtparentx=0
  \cg@gtparenty=0
  \cg@gtxadjustment=0
  \newif\ifcg@gtleft
  \newif\ifcg@gtroot
  \cg@gtroottrue
  \def\cg@node@psep{0pt}
  \def\cg@node@pname{none}
%  \psset{unit=\cg@tree@unit,nrot=:U}
  \setlength\cg@gthalfunit{\cg@tree@unit/2}
}

\def\cg@gtend{@@@ENDCGTREE@@@}

\def\cg@gtprocessnode#1{\@ifnextchar[{\cg@gtprocessnode@{#1}}{\cg@gtprocessnode@{#1}[]}}
\def\cg@gtprocessnode@#1[#2]{
  \stepcounter{cg@nextnodeord}
  \setkeys[cgk]{cgnode}{arrow=\cg@tree@arrows,sep=\cg@tree@nodesep,ko=false,pass=false,name=cggt\arabic{cg@nextnodeord}}
  \ifx.\cg@gtnextchar\def\cg@node@sep{0pt}\fi   % Override default for empty entry
  \setkeys[cgk]{cgnode}{#2}
  \ifx:\cg@gtnextchar
    % Symbolic link: don't change nextx/nexty OR render the node OR try a subtree.
    \let\next=\cg@gtprocesschar
  \else
    \ifcgk@cgnode@ko
      \cg@gtx=\the\cg@gtparentx
      \cg@gty=\the\cg@gtparenty
      \advance\cg@gtx by \ifcg@gtleft-\fi \the\cg@gtxadjustment
      \advance\cg@gtx by\ifcg@gtleft -3\else 3\fi
    \else
      \cg@gtx=\the\cg@gtnextx
      \cg@gty=\the\cg@gtnexty
      \advance\cg@gtx by \ifcg@gtleft-\fi \the\cg@gtxadjustment
      \cg@gtnextx=\the\cg@gtx
      \advance\cg@gtnextx by \ifcg@gtleft -2\else 2\fi
    \fi
    \cg@gtdonode{#1}
    \let\next=\cg@gttrysubtree
    \cg@gtxadjustment=0
  \fi
  \cg@gtdoline
  \cg@gtrootfalse
  \next
}

\def\cg@gttrysubtree{
  \@ifnextchar({\cg@gtstartsubtree}{\cg@gtprocesschar}
}

\def\cg@gtstartsubtree({
  \begingroup
  \cg@gtparentx=\the\cg@gtx
  \cg@gtparenty=\the\cg@gty
  \cg@gtnextx=\the\cg@gtparentx
  \cg@gtnexty=\the\cg@gtparenty
  \advance\cg@gtnextx by -1
  \advance\cg@gtnexty by -2
  \cg@gtlefttrue
  \edef\cg@node@psep{\cg@node@sep}
  \edef\cg@node@pname{\cg@node@name}
  \cg@gtprocesschar
}

\def\cg@gtprocesschar{\cg@fletnosp\cg@gtnextchar\cg@gtprocesschar@}
\def\cg@gtprocesschar@#1{
  \ifx\cg@gtend\cg@gtnextchar
    \let\next=\relax
  \else\ifx(\cg@gtnextchar
    \errmessage{Unexpected (}
  \else\ifx|\cg@gtnextchar
    \ifcg@gtleft
      \cg@gtleftfalse
      \cg@gtnextx=\the\cg@gtparentx
      \cg@gtnexty=\the\cg@gtparenty
      \advance\cg@gtnextx by 1
      \advance\cg@gtnexty by -2
      \let\next=\cg@gtprocesschar
    \else\errmessage{Extra | in gametree description.}
    \fi
  \else\ifx+\cg@gtnextchar
    \advance\cg@gtxadjustment by 1
    \let\next=\cg@gtprocesschar
  \else\ifx-\cg@gtnextchar
    \advance\cg@gtxadjustment by -1
    \let\next=\cg@gtprocesschar
  \else\ifx)\cg@gtnextchar
    \ifcg@gtleft\errmessage{Missing | in gametree description.}
    \else
      \let\next=\cg@gtendsubtree
    \fi
  \else
    \def\next{\cg@gtprocessnode{#1}}
  \fi\fi\fi\fi\fi\fi
  \next
}

\def\cg@gtendsubtree{
  \endgroup\cg@gtprocesschar
}

\def\cg@gtcalcmax#1{
  \ifnum \the\cg@gtx < \thecg@gtxmin
    \setcounter{cg@gtxmin}{\the\cg@gtx}
  \fi
  \ifnum \the\cg@gtx > \thecg@gtxmax
    \setcounter{cg@gtxmax}{\the\cg@gtx}
  \fi
  \ifnum \the\cg@gty < \thecg@gtymin
    \setcounter{cg@gtymin}{\the\cg@gty}
  \fi
  \ifnum \the\cg@gty > \thecg@gtymax
    \setcounter{cg@gtymax}{\the\cg@gty}
  \fi
}

\def\cg@gtshow#1{
  \ifx.#1
    \coordinate (\cg@node@name) at (\the\cg@gtx,\the\cg@gty);
  \else
    \draw (\the\cg@gtx,\the\cg@gty) node (\cg@node@name) {$#1$};
  \fi
}

\def\cg@gtdrawline{{
  \ifcg@gtroot
    % Root node - do nothing
  \else
    \ifcgk@cgnode@ko
      \ifcg@gtleft
        \def\cg@connect{to[bend left]}
	  \else
	    \def\cg@connect{to[bend right]}
	  \fi
	\else
	  \def\cg@connect{--}
	\fi
	\ifcgk@cgtree@labels
	  \def\cg@linelabel{\ifcg@gtleft L\else R\fi}
	\else
	  \def\cg@linelabel{}
	\fi
	\ifcgk@cgnode@pass
	  \ifcg@gtleft
        \draw [\cg@node@arrow] (\cg@node@pname.148) + (0.1,0) node[above] {\cg@linelabel} arc (40:320:0.6 and 0.42);
      \else
        \draw [\cg@node@arrow] (\cg@node@pname.-32) + (-0.1,0) node[above] {\cg@linelabel} arc (40:320:-0.6 and -0.42);
      \fi
    \else
      \draw [\cg@node@arrow] (\cg@node@pname) \cg@connect node[above] {\cg@linelabel} (\cg@node@name);
    \fi
  \fi
 % \psset{arrows=\cg@node@arrow,nodesepA=\cg@node@psep,nodesepB=\cg@node@sep,nodesep=\cg@node@sep}
 % \ifcgk@cgnode@ko
 %   \ifcg@gtleft\def\cg@gtangle{20}\else\def\cg@gtangle{-20}\fi
 %   \ncarc[arcangleA=\cg@gtangle,arcangleB=\cg@gtangle]{\cg@node@pname}{\cg@node@name}
 % \else\ifcgk@cgnode@pass
 %   \ifcg@gtleft\def\cg@gtangle{90}\else\def\cg@gtangle{-90}\fi
 %   \nccircle[angleA=\cg@gtangle]{\cg@node@pname}{\cg@gthalfunit}
 % \else
 %   \ncline{\cg@node@pname}{\cg@node@name}
 % \fi\fi
 % \ifcgk@cgtree@labels
 %   \naput{\ifcg@gtleft L\else R\fi}
 % \fi
}}
