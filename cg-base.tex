\ProvidesFile{cg-base.tex}

%%%%%%%%%%%%%%%%%%%%%%%%%%%%%%%%%%%%%%%%%%%%%%%%%%%%%%%%%%%%%%
%%                                                          %%
%%              Raymond Chen's Game Builder                 %%
%%                  (somewhat altered)                      %%
%%                                                          %%
%%%%%%%%%%%%%%%%%%%%%%%%%%%%%%%%%%%%%%%%%%%%%%%%%%%%%%%%%%%%%%

% The \combgame macro takes as its argument the description of a game.
% It measures the size of the contents and scales all the slashes
% inside to the proper size.
%
% This makes eight passes over the input argument.
%
% For each of the four typestyles...
%
%   Typeset the argument in that style.
%   Set \cg@gamebox to an empty box with the height and depth of the game.
%       (During this first pass, \ifcg@drawbars is \iffalse.)
%   If we are in display style, set \cg@extension to the size of the
%       delimiter extension.  (If not, then set it to zero.)
%   Typeset the game proper, setting \ifcg@drawbars to \iftrue.
%
% This macro requires cooperation from the \cgslash, \cgsslash, etc. macros.

\newmuskip\cgslashmuskip
\cgslashmuskip=-2mu
\newlength\cgslashextension
\setlength\cgslashextension{1.5pt}

\def\cg@slashskip{\mskip\cgslashmuskip}

\newif\ifcg@star
\newif\ifcg@building\cg@buildingfalse
\newif\ifcg@drawbars\cg@drawbarsfalse
\newbox\cg@gamebox\setbox\cg@gamebox=\null
\newdimen\cg@extension
\newcounter{cg@readbarctr}

\newcommand\combgame{%
  \begingroup%
  \catcode`\|=\active%
  \cg@definebars%
  \@ifstar{\cg@startrue\cg@combgame}{\cg@starfalse\cg@combgame}%
}

\def\cg@combgame#1{%
  \mathpalette\cg@buildgame{#1}%
  \endgroup%
}

\def\cg@buildgame#1#2{%
    \begingroup%
    \cg@buildingtrue%
    \ifx#1\displaystyle%
      \ifcg@star%
        \cg@extension=0pt%
      \else%
        \cg@extension=\the\cgslashextension%
      \fi%
    \fi%
    \cg@drawbarsfalse%
    \setbox\z@\hbox{$#1#2$}%
    \ht\cg@gamebox\ht\z@%
    \dp\cg@gamebox\dp\z@%
    \cg@drawbarstrue%
    #1{#2}%
    \cg@buildingfalse%
    \endgroup%
}

\begingroup
\catcode`\|=\active
\gdef\cg@definebars{%
  \def|{\setcounter{cg@readbarctr}{1}\cg@readbars}%
  \def\{{\left\lbrace}
  \def\}{\right\rbrace}
}
\gdef\cg@readbars{%
  \@ifnextchar|{\stepcounter{cg@readbarctr}\cg@readnextbar}{\cgslashes{\thecg@readbarctr}}%
}
\gdef\cg@readnextbar|{\cg@readbars}
\endgroup

\newcommand\cgslash{\cgslashes{1}}
\newcommand\cgsslash{\cgslashes{2}}

\newcommand\cgslashes[2][]{%
  \ifcg@building
    \ifcg@drawbars
        \mathrel
        {\hbox{$%
    %% This part is deadly because \left and \right create implicit
    %% grouping.
        \toks\z@{}%
        \cg@loopmbars{#2}{\global\toks\z@\expandafter{\the\toks\z@\cg@mbarshelper}}%
        \def\cg@mbarshelper{\left\vert\mskip\cgslashmuskip}\the\toks\z@
            \mskip-\cgslashmuskip
            \@tempcnta#2\relax \advance\@tempcnta\m@ne
            \raise\@tempcnta\cg@extension\copy\cg@gamebox
            \lower\@tempcnta\cg@extension\copy\cg@gamebox
        \gdef\cg@mbarshelper{\right.}\the\toks\z@\n@space$}}%
    \else\mathrel\vert\fi
  \else
    \cg@badbar
  \fi
}

\def\cg@loopmbars#1#2{
    \@tempcnta#1\relax
    \loop
        #2%
        \advance\@tempcnta\m@ne
        \ifnum\@tempcnta>\z@
    \repeat
}

\def\cg@badbar{\errmessage{cgslash commands may only be used inside a combgame command.}}

%%%%%%%%%%%%%%%%%%%%%%%%%%%%%%%%%%%%%%%%%%%%%%%%%%%%%%%%%%%%%%
%%                                                          %%
%%                      Thermographs                        %%
%%                                                          %%
%%%%%%%%%%%%%%%%%%%%%%%%%%%%%%%%%%%%%%%%%%%%%%%%%%%%%%%%%%%%%%

\define@key[cgk]{thermoplot}{left}{\def\cg@plot@left{#1}}
\define@key[cgk]{thermoplot}{right}{\def\cg@plot@right{#1}}
\define@key[cgk]{thermoplot}{top}{\def\cg@plot@top{#1}}
\define@key[cgk]{thermoplot}{bottom}{\def\cg@plot@bottom{#1}}
\define@key[cgk]{thermoplot}{tickspacing}{\def\cg@plot@tickspacing{#1}}
\define@key[cgk]{thermoplot}{scale}{\def\cg@plot@scale{#1}}
\define@key[cgk]{thermoplot}{axescolor}{\def\cg@plot@axescolor{#1}}
\define@key[cgk]{thermoplot}{clipmargin}{\def\cg@plot@clipmargin{#1}}
\define@key[cgk]{thermoplot}{baseline}{\def\cg@plot@baseline{#1}}
\define@key[cgk]{thermograph}{linecolor}{\def\cg@therm@linecolor{#1}}
\define@key[cgk]{thermograph}{linestyle}{\def\cg@therm@linestyle{#1}}
\define@key[cgk]{thermograph}{displ}{\def\cg@therm@displ{#1}}

\newlength\cg@xorigin
\newlength\cg@yorigin
\newlength\cg@xsize
\newlength\cg@ysize
\newlength\cg@vcoord
\newlength\cg@tcoord

\newenvironment{thermoplot}[1][]{%
  \setkeys[cgk]{thermoplot}{left=2,right=-2,top=4,bottom=0,scale=1cm,axescolor=gray,clipmargin=6pt,baseline=0.0,tickspacing=1}%
  \setkeys[cgk]{thermoplot}{#1}%
  \setlength\cg@xorigin{\cg@plot@scale*\cg@plot@left}%
  \setlength\cg@yorigin{-\cg@plot@scale*\cg@plot@bottom}%
  \setlength\cg@xsize{\cg@plot@scale*\cg@plot@left-\cg@plot@scale*\cg@plot@right}%
  \setlength\cg@ysize{\cg@plot@scale*\cg@plot@top-\cg@plot@scale*\cg@plot@bottom}%
  \psset{xunit=-\cg@plot@scale,yunit=\cg@plot@scale}%
  \begin{pspicture}(\the\cg@xsize,\the\cg@ysize)%
  {\color{\cg@plot@axescolor}%
   \renewcommand\pshlabel[1]{##1}%
   \psaxes[linecolor=\cg@plot@axescolor,Dx=-\cg@plot@tickspacing,Dy=\cg@plot@tickspacing,subticks=\cg@plot@tickspacing](\the\cg@xorigin,\the\cg@yorigin)(0,0)(\the\cg@xsize,\the\cg@ysize)}%
  \addtolength{\cg@xsize}{\cg@plot@clipmargin}\addtolength{\cg@ysize}{\cg@plot@clipmargin}%
  \psclip{\psframe[linecolor=white](-\cg@plot@clipmargin,-\cg@plot@clipmargin)(\the\cg@xsize,\the\cg@ysize)}%
}{%
  \endpsclip
  \end{pspicture}%
}%
\newcommand\trajectory[1][]{%
  \setkeys[cgk]{thermograph}{linecolor=black,linestyle=solid,displ=0pt}%
  \setkeys[cgk]{thermograph}{#1}%
  \def\cg@coords{{<-}}%
  \cg@readcoordsfortrajectory%
}
\newcommand\thermograph[2][]{%
  \setkeys[cgk]{thermograph}{linecolor=black,linestyle=solid,displ=0pt}%
  \setkeys[cgk]{thermograph}{#1}%
  \def\cg@therm@mast{#2}%
  \cg@starttrajectory%
  }
\def\cg@starttrajectory{%
  \cg@setvcoord{\cg@therm@mast}%
  \cg@settcoord{\cg@plot@top}%
  \addtolength{\cg@vcoord}{\cg@therm@displ}%
  \edef\cg@coords{{<-}(\the\cg@vcoord,\the\cg@tcoord)}%
  \cg@readcoordsfortrajectory%
}
\def\cg@readcoordsfortrajectory{%
  \@ifnextchar(\cg@readcoord%
    {\expandafter\cg@trajectoryline\cg@coords\cg@readnexttrajectory}%
}
\def\cg@trajectoryline{\psline[linecolor=\cg@therm@linecolor,linestyle=\cg@therm@linestyle]}
\def\cg@readcoord(#1,#2){%
  \cg@setvcoord{#1}%
  \cg@settcoord{#2}%
  \addtolength{\cg@vcoord}{\cg@therm@displ}%
  \edef\cg@coords{\cg@coords(\the\cg@vcoord,\the\cg@tcoord)}%
  \cg@readcoordsfortrajectory%
}
\def\cg@settcoord#1{%
  \setlength\cg@tcoord{\cg@plot@scale*#1-\cg@plot@scale*\cg@plot@bottom}%
}
\def\cg@setvcoord#1{%
  \setlength\cg@vcoord{\cg@plot@scale*\cg@plot@left-\cg@plot@scale*#1}%
}
\def\cg@readnexttrajectory{\@ifnextchar;{\cg@eatsemicolon}{}}
\def\cg@eatsemicolon;{\cg@starttrajectory}

