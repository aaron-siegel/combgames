\documentclass{article}

\usepackage{bigstrut,array,float,fancyvrb,amsmath,amssymb,marvosym,verbatim,combgames}
\usepackage{hyperref}

\newcommand\cn[1]{\texttt{\string#1}}
\newcommand\ds[1]{$#1$ & \cn{#1}}
\newcommand\combgames{\textsf{combgames}}

\newenvironment{VExample}
  {\VerbatimEnvironment\begin{VerbatimOut}{example.out}}
  {\end{VerbatimOut}
   \VerbatimInput{example.out}
   \medskip\par
   \begin{center}$\input example.out $\end{center}}

\newenvironment{HExample}
  {\VerbatimEnvironment\begin{VerbatimOut}{example.out}}
  {\end{VerbatimOut}
   \begin{center}
   \begin{minipage}{2.25in}
   \BVerbatimInput{example.out}
   \end{minipage}
   \begin{minipage}{2.25in}
   \centering$\input example.out $
   \end{minipage}
   \end{center}}

\title{The \combgames{} Package}
\author{Aaron N. Siegel\\\texttt{aaron.n.siegel@gmail.com} \and
        David Wolfe\\\texttt{davidgameswolfe@gmail.com}}

\begin{document}

\maketitle

\tableofcontents

\section{The Symbol Font}

\combgames{} defines the following symbols.

\begin{table}[H]
\centering
\begin{tabular}{clclcl}
\ds{\cgup} & \ds{\cgdown} & \ds{\cgstar} \bigstrut\\
\ds{\cgdoubleup} & \ds{\cgdoubledown} & \ds{\cgneg} \bigstrut\\
\ds{\cgtripleup} & \ds{\cgtripledown} & \ds{\cgfarstar} \bigstrut\\
\ds{\cgquadup} & \ds{\cgquaddown} & \ds{\cgsunny} \bigstrut\\
\ds{\cgtiny} & \ds{\cgminy} & \ds{\cgloony} \bigstrut\\
\ds{\cgko} & \ds{\cgKo} \bigstrut\\
\ds{\cgkobar} & \ds{\cgKobar} \bigstrut
\end{tabular}
\caption{\combgames{} ordinary symbols.}
\end{table}

\begin{table}[H]
\centering
\begin{tabular}{clclcl}
\ds{\cglfuz} & \ds{\cggfuz} & \ds{\cgfuzzy} \bigstrut\\
\ds{\cgupsum} & \ds{\cgdownsum} & \ds{\cgnmultiply} \bigstrut
\end{tabular}
\caption{\combgames{} binary relations.}
\end{table}

A few notes.

\begin{itemize}
\item Usage of \cn{\cgnmultiply} for Norton multiplication has been largely supplanted by a simple \cn{\cdot}.  Compare:
\[G \cgnmultiply \cgup \qquad\qquad \textrm{versus} \qquad\qquad G \cdot \cgup\]
\item This author dislikes the \cn{\cgfuzzy} symbol, precisely because there are already too many slashes in CGT.  A reasonable alternative is \cn{\not}\cn{\gtrless}, which is available in the \textsf{amssymb} package.  Compare:
\[G \cgfuzzy H \qquad\qquad \textrm{versus} \qquad\qquad G \not\gtrless H\]
\item The \textsf{marvosym} package provides a \cn{\Moon} symbol that can be used as an alternative to \cn{\cgloony}:
\[\cgloony \qquad\qquad \textrm{versus} \qquad\qquad \textrm{\Moon}\]
\end{itemize}

\section{Braces-and-Slashes Notation}

\combgames{} also has a powerful facility for typesetting games using braces-and-slashes notation.  The basic command is \cn{\combgame}.  All braces and slashes used within the command will be properly spaced, and in displayed equations they'll be sized to fit the surrounding material.  Some examples:

\begin{table}[H]
{
\centering
\begin{tabular}{@{}l@{\hspace{0.3in}}c@{}}
%%%%%
\begin{verb}
\combgame{2||1|0}
\end{verb}
&
$\displaystyle \combgame{2||1|0}$
\bigskip \\
%%%%%
\begin{verb}
\combgame{\{3,\{4||2|1\}|||0||||-8\}}
\end{verb} &
$\displaystyle \combgame{\{3,\{4||2|1\}|||0||||-8\}}$
\bigskip
\end{tabular}
%%%%%
}

\begin{verb}
\int^T G = \combgame{\{T + \int^T G^L | -T + \int^T G^R\}}
\end{verb}
\bigskip

\hfill$\displaystyle \int^T G = \combgame{\{T + \int^T G^L | -T + \int^T G^R\}}$

\caption{Example usage of \cn{\combgame}.}
\end{table}

\begin{itemize}
\item There is also a starred form, \cn{\combgame*}, that suppresses growth of the vertical bars:
\begin{center}
\begin{tabular}{@{}l@{\hspace{0.3in}}c@{}}
%%%%%
\begin{verb}
\combgame{2|1||0|||-1||||-2}
\end{verb}
&
$\displaystyle \combgame{2|1||0|||-1||||-2}$
\bigskip \\
%%%%%
\begin{verb}
\combgame*{2|1||0|||-1||||-2}
\end{verb} &
$\displaystyle \combgame*{2|1||0|||-1||||-2}$
\bigskip
\end{tabular}
%%%%%
\end{center}

\item You should use \cn{\combgame} whenever appropriate, even for simple expressions, since it will typeset the result much more cleanly than otherwise.  Compare the following two examples.

\begin{center}
\begin{tabular}{l@{\hspace{1cm}}l@{\hspace{1cm}}l}
\begin{verb}
\combgame*{\{2||-1|-3\}}
\end{verb}
&
$\combgame*{\{2||-1|-3\}}$
&
Beautiful!
\bigskip \\
\begin{verb}
\{2||-1|-3\}
\end{verb}
&
$\{2||-1|-3\}$
&
Hideous!
\end{tabular}
\end{center}

\item Within the \texttt{combgame} argument, the brace commands \cn{\{} and \cn{\}} are redefined to mean \cn{\left}\cn{\{} and \cn{\right}\cn{\}}.  The old commands are still available as \cn{\lbrace} and \cn{\rbrace}.
\item Due to the way \TeX{} processes command inputs, the slash notation will not function correctly if used inside a macro.  If you wish to define macros that refer to \cn{\combgame}, there is an alternative command \verb|\cgslashes{n}| for an $n$-tuple slash.  The commands \cn{\cgslash} and \cn{\cgsslash} are shorthand for $n = 1$ and $2$, respectively.  Here's an example:

\begin{verbatim}
\newcommand\threeswitch[3]
  {\combgame{\{#1 \cgsslash #2 \cgslash #3\}}}
\end{verbatim}

Then \verb|\threeswitch{2}{1}{0}| would typeset

\[\combgame{\{2||1|0\}}\]

\item You can control the growth of the slashes by setting \cn{\cgslashextension}.  The default is \texttt{1.5pt}.  For example:
\begin{center}
\begin{tabular}{@{}l@{\hspace{0.3in}}c@{}}
%%%%%
\begin{verb}
\combgame{2|1||0|||-1||||-2}
\end{verb}
&
$\displaystyle \combgame{2|1||0|||-1||||-2}$
\bigskip \\
%%%%%
\begin{minipage}{2.75in}
\begin{verbatim}
\setlength\cgslashextension{3pt}
\combgame{2|1||0|||-1||||-2}
\end{verbatim}
\end{minipage} &
$\displaystyle \setlength\cgslashextension{3pt} \combgame{2|1||0|||-1||||-2}$
\bigskip
\end{tabular}
%%%%%
\end{center}
\end{itemize}

\section{Game Trees}

You can typeset game trees easily with the powerful \cn{\cgtree} command.  Trees should be placed inside a \texttt{pspicture} environment and can coexist with other pstricks objects.  Here's a simple example:

\begin{VExample}
\begin{pspicture}(5,4)
\put(1,4){\cgtree{
  {\cgup^2} (0 | {\cgdown\cgstar} (0 | 0 \cgstar))
}}
\end{pspicture}
\end{VExample}

The general syntax of a \cn{\cgtree} argument is \verb|\cgtree{node}| where \texttt{node} has the following specification:

\begin{verbatim}
node    ::=   label (subtree)?
            | special

subtree ::= '(' (node)* '|' (node)* ')'
\end{verbatim}

\texttt{label} may be either a single token, or a more complicated expression delineated by braces.  The two nodelists in the subtree expression are typeset as left and right options of the parent node.  Left options are laid out right-to-left; Right options left-to-right; always starting at the parent node.  If the label is \texttt{.} then a node will be created with no label.  \texttt{special} may be one of the following:

\begin{itemize}
\item \texttt{+} increases the space between options
\item \texttt{-} decreases the space between options
\item \texttt{:} creates a symbolic link (described below)
\end{itemize}

The \texttt{cgtree} command takes several options:

\begin{itemize}
\item \texttt{arrows} specifies the arrowheads used for tree edges, as in pstricks.  Example: \texttt{arrows=->}
\item \texttt{unit} specifies the scale for drawing the tree, e.g., \texttt{unit=.5cm} (default: 1cm)
\item \texttt{nodesep} specifies the separation between edges and node boundaries.  Default is \texttt{nodesep=.75ex}
\end{itemize}

In addition, each \emph{node} may have several options.  Node options should be written in brackets immediately following the node label (and before the associated subtree, if one exists).

\begin{itemize}
\item \texttt{arrow} specifies the arrowhead for the edge \emph{to} this node
\item \texttt{sep} specifies the separation for this node
\item \texttt{name} gives a name for this node.  \emph{Names must consist only of letters and numbers}.  They can be used in symbolic links (see below) or elsewhere in the \texttt{pspicture} environment.
\item \texttt{ko} draws the edge \emph{to} this node as a ko.
\end{itemize}

If a \texttt{:} is specified instead of a node, a symbolic link is created.  The \texttt{:} must be followed by a list of options in brackets, which \emph{must} include the \texttt{name} option.  An edge will then be drawn to the previously named node.  Be careful, as this will \emph{not} take into account whether the edge points left or right!

\begin{VExample}
\begin{pspicture}(6,6)
\put(1,6){\cgtree{
 G ( . ( | {K_1} [name=koroot] (1 | {K_2} [ko,arrow=<->] (|0) ) )
   | . (:[name=koroot] | )
   )}}
\end{pspicture}
\end{VExample}

\bigskip

One last example: the following mess typesets a pretty figure from Bill Fraser's thesis.

\begin{VExample}
\psset{unit=.6cm}
\begin{pspicture}(19,6)
\put(5,6){\cgtree[unit=.6cm]{
  {D_1}[name=D1] ( U (T[ko,name=T] (4|) | 3[name=three])
  {D^L}[ko] (:[name=T] | +++++++{D_2}[ko] (2 |
    W(.(2|-.[ko,name=WLR](|1)) | -.[ko] (:[name=WLR,sep=0pt]|{-10}))
    +++{D^R}[ko](X :[ko,name=D1]|{-11})
  )) | V (:[name=three] | .[ko](.(-.[ko](2|)|1)|{-10})))
}}
\end{pspicture}
\end{VExample}

\section{Game Boards}

There is also an extensible facility for typesetting grid-based games.  The following examples illustrate usage for several of the built-in games.

\begin{VExample}
\Domineering{ooo\\oo\\} = \combgame{
  \{ \Domineering{^oo\\vo\\},
     \Domineering{o^o\\ov\\}
   | \Domineering{<>o\\oo\\},
     \Domineering{o<>\\oo\\},
     \Domineering{ooo\\<>\\}   \}}
\end{VExample}

\bigskip

\begin{HExample}
\clobber{xxxo\\o.ox\\xoxo}
\end{HExample}

\medskip

\begin{HExample}
\Clobber{xxxo\\o.ox\\xoxo}
\end{HExample}

\medskip

\begin{HExample}
\DotsAndBoxes[unit=.5cm]{
  o+o.o.o-o-o \\
  |.|.|.|.*.| \\
  o.o.o.o*o*o \\
  |.|.|.|.|.| \\
  o-o.o.o*o*o \\
  |A|.|.|.|.| \\
  o-o.o.o*o*o}
\end{HExample}

\medskip

\begin{HExample}
\Amazons[unit=.5cm]{..x\\.L.\\x.R}
\end{HExample}

\bigskip

You can declare new games easily with the \cn{\newgridgame} command.  The syntax is very simple.  The command argument consists of a list of allowed grid characters together with instructions for typesetting them.  For example, the \cn{\clobber} command is declared as follows.

\begin{verbatim}
\newgridgame[unit=.3cm]{clobber}{
  X {\ClobberX}
  O {\ClobberO}
  x {\ClobberX}
  o {\ClobberO}
}
\end{verbatim}

Here \cn{\ClobberX} and \cn{\ClobberO} are primitive commands for rendering clobber symbols.

\section{Thermographs}

\combgames{} provides a \texttt{thermoplot} environment that can be used to draw an arbitrary number of thermographs together on the same plot.  For example:

\begin{VExample}
\begin{thermoplot}[left=5/2,right=-5/2]
\thermograph{0}(0,3)(2,1)(2,-1);(0,3)(-1,2)(-1,1)(-3,-1)
\end{thermoplot}
\end{VExample}

\bigskip

Coordinates must be specified as \emph{rational numbers in the usual (CGT) coordinate space}.  The \cn{\thermograph} command is followed by an argument specifying the mast value.  This is followed by any number of semicolon-separated trajectories, listing the $(\textit{v},\textit{t})$-coordinates of each critical point on the trajectory.

Multiple thermographs can be placed on the same plot:

\begin{VExample}
\begin{thermoplot}[left=6,right=-19,top=12,scale=0.45cm]
\thermograph[linecolor=blue]{1/2}
  (1/2,13/2)(2,5)(2,2)(8,-1);
  (1/2,13/2)(2,5)(2,2)(-1,-1)
\thermograph[linecolor=red]{-19/4}
  (-19/4,19/4)(1,-1);
  (-19/4,19/4)(-21,-1)
\end{thermoplot}
\end{VExample}

\bigskip

Note how, in these examples, each trajectory is carried down to $-1$ instead of~$0$.  This generates the ``hooks'' that extend below $t = 0$ on the thermographs.

You can also place individual trajectories instead of full thermographs.  This can be useful if the mast has nonzero slope.  For example:

\begin{VExample}
\begin{thermoplot}[left=1,right=-4]
\trajectory[linecolor=lightgray](1,4)(-1,2)(-1,1)(-3,-1)
\thermograph{-3}(-3,2)(-2,1)(-2,-1);(-3,2)(-4,1)(-4,-1)
\end{thermoplot}
\end{VExample}

\end{document}
